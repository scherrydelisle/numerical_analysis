
% This LaTeX was auto-generated from MATLAB code.
% To make changes, update the MATLAB code and republish this document.

\documentclass{article}
\usepackage{graphicx}
\usepackage{color}

\sloppy
\definecolor{lightgray}{gray}{0.5}
\setlength{\parindent}{0pt}

\begin{document}

    
    

\section*{Test\_Newton\_2a.m}

\begin{par}
The script Test\_Secant\_1b.m tests the function newton\_2a() with a = 3, x\_0 = 0, MAX\_N = 20, and TOL =
\end{par} \vspace{1em}
\begin{par}
AUTHOR: Simone Cherry-Delisle UCID: 10144438 COURSE: MATH 391 ASSIGNMENT: Assignment 3 Q 2a
\end{par} \vspace{1em}
\begin{verbatim}
x_0 = 0;
MAX_N = 20;
TOL = 1.0e-10;

x_sol = newton_2a(x_0, MAX_N, TOL);
fprintf('The Newton approx for the root of f(x)=0 is x=%.20f\n',x_sol);
\end{verbatim}

        \color{lightgray} \begin{verbatim}For i=1, approx soln is x=0.58197670686932634343
   abs(x - x_exact) is 0.41802329313067365657
For i=2, approx soln is x=0.80550807513701938589
   abs(x - x_exact) is 0.19449192486298064186
For i=3, approx soln is x=0.90590431107900726282
   abs(x - x_exact) is 0.09409568892099277881
For i=4, approx soln is x=0.95368987990567033464
   abs(x - x_exact) is 0.04631012009432970700
For i=5, approx soln is x=0.97702365250032996435
   abs(x - x_exact) is 0.02297634749967001830
For i=6, approx soln is x=0.98855581857512875299
   abs(x - x_exact) is 0.01144418142487121579
For i=7, approx soln is x=0.99428882337110502387
   abs(x - x_exact) is 0.00571117662889492929
For i=8, approx soln is x=0.99714712981226283350
   abs(x - x_exact) is 0.00285287018773719816
For i=9, approx soln is x=0.99857424314502640783
   abs(x - x_exact) is 0.00142575685497355119
For i=10, approx soln is x=0.99928729097098101253
   abs(x - x_exact) is 0.00071270902901900731
For i=11, approx soln is x=0.99964368781484891358
   abs(x - x_exact) is 0.00035631218515113066
For i=12, approx soln is x=0.99982185448697957764
   abs(x - x_exact) is 0.00017814551302039534
For i=13, approx soln is x=0.99991092988752328097
   abs(x - x_exact) is 0.00008907011247674816
For i=14, approx soln is x=0.99995546560364823030
   abs(x - x_exact) is 0.00004453439635173255
For i=15, approx soln is x=0.99997773296462588721
   abs(x - x_exact) is 0.00002226703537407097
For i=16, approx soln is x=0.99998886651868279518
   abs(x - x_exact) is 0.00001113348131715207
For i=17, approx soln is x=0.99999443325977388497
   abs(x - x_exact) is 0.00000556674022611841
For i=18, approx soln is x=0.99999721661267515493
   abs(x - x_exact) is 0.00000278338732483014
For i=19, approx soln is x=0.99999860826739517261
   abs(x - x_exact) is 0.00000139173260487450
For i=20, approx soln is x=0.99999930405468506756
   abs(x - x_exact) is 0.00000069594531493820

The Newton approx for the root of f(x)=0 is x=0.99999930405468506756
\end{verbatim} \color{black}
    


\end{document}
    
