
% This LaTeX was auto-generated from MATLAB code.
% To make changes, update the MATLAB code and republish this document.

\documentclass{article}
\usepackage{graphicx}
\usepackage{color}

\sloppy
\definecolor{lightgray}{gray}{0.5}
\setlength{\parindent}{0pt}

\begin{document}

    
    \begin{verbatim}
% Test_Secant_1b.m
%
% The script Test_Secant_1b.m tests the function secant_1b() with a = 3,
% x_0 = 1, x_1 = 2, and MAX_N = 10
%
% AUTHOR: Simone Cherry-Delisle
% UCID: 10144438
% COURSE: MATH 391
% ASSIGNMENT: Assignment 3 Q 1b

a = 3;
x_0 = 1;
x_1 = 2;
MAX_N = 10;
TOL = 1.0e-13;

x_sol = secant_1b(a,x_0,x_1,MAX_N, TOL);
fprintf('The secant approx for the root of f(x)=0 is x=%.11f\n',x_sol);
\end{verbatim}

        \color{lightgray} \begin{verbatim}For i=1, approx soln is x=1.00000000000
For i=2, approx soln is x=1.28571428571
   abs(x - x_exact) is 0.44633652185
For i=3, approx soln is x=1.39205955335
   abs(x - x_exact) is 0.33999125422
For i=4, approx soln is x=1.44826535047
   abs(x - x_exact) is 0.28378545710
For i=5, approx soln is x=1.44203589210
   abs(x - x_exact) is 0.29001491547
For i=6, approx soln is x=1.44224868142
   abs(x - x_exact) is 0.28980212615
For i=7, approx soln is x=1.44224957044
   abs(x - x_exact) is 0.28980123713
For i=8, approx soln is x=1.44224957031
   abs(x - x_exact) is 0.28980123726
For i=9, approx soln is x=1.44224957031
   abs(x - x_exact) is 0.28980123726

The secant approx for the root of f(x)=0 is x=1.44224957031
\end{verbatim} \color{black}
    


\end{document}
    
